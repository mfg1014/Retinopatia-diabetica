\capitulo{4}{Técnicas y herramientas}

\section{Metodologías}
\textbf{SCRUM}\newline
 SCRUM es una metodología ágil basada en una estrategia continua e incremental, cuyo objetivo es proporcionar un producto funcional al final de cada periodo de trabajo planeado (\textit{sprint}), haciendo reuniones diarias y antes y después de cada \textit{sprint} otras reuniones donde se explican los problemas que se han tenido y como se va a planear el siguiente.

El impedimento más notorio de esta metodología, es que hay un equipo de personas entre las que se encuentra el \textit{product owner}, el \textit{SCRUM master} y el equipo de desarrollo. Además el equipo se recomienda ser de entre 3 a 9 personas para un buen desarrollo, por tanto, en este trabajo ha sido complicado ir haciendo todas las acciones que se piden en la metodología SCRUM.

\textbf{GitFlow}\newline
GitFlow es un flujo de trabajo, en el cual se ramifica el proyecto en \textit{branches} donde cada una, contiene una parte del proyecto; de esta forma, se puede dejar una parte funcional sin modificar que seria la rama principal, y otras ramas, donde se van realizando los cambios, cuando en estas ramas, se termina la tarea que se esta realizando, haciendo un producto funcional, se realiza una operación de \textit{pull request} para combinar ambas ramas. Esta operación es aceptada o denegada por el equipo de desarrollo que no haya participado en la realización de esta rama.

En mi proyecto, se han creado tres \textit{branches}, que son:
\begin{itemize}
    \item \textit{main}: es la rama principal, donde se alberga la versión estable del proyecto.
    \item boceto: es una rama secundaria, donde se realizan los cambios que se están realizando en la aplicación móvil.
    \item latex: es la rama secundaria donde se realizan los cambios que se están realizando en el documento LaTeX.
\end{itemize}




