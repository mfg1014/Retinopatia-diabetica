\capitulo{1}{Introducción}

La discapacidad visual afecta a más de 2200 millones de personas en el mundo, siendo las principales causas de la perdida de visión: degeneración macular relacionada con la edad, cataratas, retinopatía diabética, glaucoma y errores de refracción no corregidos.\cite{oms-ceguera}


Entre las personas con discapacidad visual, hay al menos 1000 millones de personas que tienen un deterioro moderado o grave de la visión.\cite{oms-ceguera} 

En Estados Unidos, cada año la retinopatía diabética suma un 12\% de nuevos casos de ceguera.

La retinopatía diabética es una complicación de la diabetes, que afecta al sistema ocular del ser humano. Es causada por el daño a los vasos sanguíneos de la retina a lo largo del tiempo. Esta anomalía, encabeza las causas de ceguera en los países desarrollados. Según la Organización Mundial de la Salud, hasta 1 millón de personas tienen ceguera debido a la diabetes. \cite{oms-diabetes}
Hay varios grados de la patología de la retinopatía diabética entre los que se encontrarían por orden de menor a mayor peligrosidad: NPDR (Non-proloferative diabetic retinopathy), este grado se caracteriza por la ausencia de retinopatía; como siguiente grado se encontraría NPDR leve, donde los vasos sanguíneos empiezan a debilitarse, creando protuberancias llamadas micro-aneurismas; en tercer se encuentra NPDR moderada, donde los vasos sanguíneos se siguen debilitando, se producen más hemorragias, pudiendo provocar visión borrosa; siguiendo por NPDR severo, en esta etapa, los vasos sanguíneos están dañados, causando falta de oxigeno en la retina y en la formación de nuevos vasos; por ultimo, se encuentra retinopatía diabética proliferativa, donde los vasos sanguíneos anormales que crecen en la retina y en el vítreo. Estos vasos pueden sangrar y provocar desprendimiento de retina, provocando la perdida de visión. \cite{grados-retinopatia}

Para determinar el grado que tiene un paciente de retinopatía diabética, se tiene que desplazar al hospital donde le hacen una imagen de la retina con un retinógrafo. Pero, para la sanidad publica tiene un coste elevado, y provoca que el paciente se tenga que desplazar a un centro hospitalario. 

Como solución a este problema, se propone realizar una aplicación móvil, que permita al paciente o al médico hacer una foto de la retina, utilizando una lente para el dispositivo móvil. Además, para facilitar la labor del médico, esta aplicación móvil, proporciona redes neuronales ya entrenadas, con los que se obtendría un primer análisis, proporcionando un estudio más intenso a aquellas personas que hayan dado algún grado de retinopatía diabética. 

De esta forma, al instalar la aplicación RetinAI creada con el objetivo de ser útil para el sistema sanitario, usando una interfaz sencilla para médicos sin experiencia previa en aplicaciones Android. Además, la aplicación facilita la creación de un informe médico ya que interpreta los resultados obtenidos de las imágenes, dando prioridad a aquellos casos que tengan más gravedad; y descartando aquellos casos que no tengan retinopatía diabética. Evitando una lentitud en el sistema sanitario debido a pacientes que no tienen esta complicación. 
Permitiendo a los médicos centrar su atención en aquellos pacientes que realmente tienen retinopatía diabética.