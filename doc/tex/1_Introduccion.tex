\capitulo{1}{Introducción}

Durante los últimos años, se ha aumentado exponencialmente el uso de los dispositivos móviles, este aumento se debe principalmente a la comodidad que proporcionan respecto a otras tecnologías parecidas; siendo principalmente útiles para el uso de las redes sociales. 

Este aumento, tiene como consecuencia el fomento del desarrollo de aplicaciones móviles, puesto que, sin aplicaciones para estos dispositivos, no hubiesen tenido tanto auge. Estas aplicaciones están destinadas a dos sistemas operativos principalmente, que son Android e iOS; donde el 67,56\% de los usuarios prefieren Android y el 31.6\% iOS~\cite{mobile-os-market-share}.

Los móviles, no solo han sufrido un desarrollo significativo en lo referente a software, sino en hardware también, donde se han aumentado las características de procesador, cámara, RAM,... Y con ello, los desarrolladores
buscan crear aplicación que hagan uso de este hardware permitiendo muchas más funcionalidades. Una de estas funcionalidades es la integración de inteligencias artificiales que en los últimos años han reducido considerablemente los recursos y tiempo que necesitan para ejecutarse. Esta funcionalidad se ha convertido en una parte fundamental de la experiencia del usuario. Por ejemplo, aplicaciones de reconocimiento de voz como los asistentes virtuales (que utilizan algoritmos de procesamiento del lenguaje), también existen aplicaciones de reconocimiento facial para desbloquear el dispositivo (que se encuentra en el área de visión artificial) y el último ejemplo, son traducciones en tiempo real que facilitan la comunicación entre idiomas. Estos distintos usos que se dan a la inteligencia artificial están transformando la forma de interactuar con las aplicaciones móviles, brindando una experiencia más fluida y personalizada al usuario.

Por otro lado, la salud y bienestar han sido temas importantes para el ser humano; y por lo tanto, también se han hecho aplicaciones móviles con este objetivo. Como son aplicaciones del SACYL~\cite{sacyl-app}, de Adeslas~\cite{adeslas}, de la Organización Mundial de la salud~\cite{who-infoapp} y de Google~\cite{google-fitness-app} entre otras.


Además de las aplicaciones mencionadas anteriormente, también se han desarrollado otras más específicas destinadas a ayudar a las personas con discapacidad visual, la cual afecta a más de 2.200 millones de personas en el mundo. Las principales causas de pérdida de visión son la degeneración macular relacionada con la edad, las cataratas, la retinopatía diabética, el glaucoma y errores de refracción no corregidos. Unos 1.000 millones de personas tienen un deterioro moderado o grave de la visión\cite{oms-ceguera}.

La retinopatía diabética~\cite{enfermedad-ocular-diabetica} es una complicación de la diabetes, que afecta al sistema ocular del ser humano. Es causada por la inflamación, escape o cierre de los vasos sanguíneos de la retina, pudiendo desarrollarse nuevos vasos sanguíneos a lo largo del tiempo. Esta anomalía, encabeza las causas de ceguera en los países desarrollados. Según la Organización Mundial de la Salud, hasta 1 millón de personas tienen ceguera debido a la diabetes~\cite{oms-diabetes}. 
En Estados Unidos, cada año la retinopatía diabética suma un 12\% de nuevos casos de ceguera~\cite{porcentaje-afectados-eeuu}.
Hay varios grados de la patología de la retinopatía diabética entre los que se encontrarían por orden de menor a mayor peligrosidad~\cite{grados-retinopatia}: 
\begin{itemize}
    \item NPDR (Non-proloferative diabetic retinopathy), este grado se caracteriza por la ausencia de retinopatía;
    \item NPDR leve, donde los vasos sanguíneos empiezan a debilitarse, creando protuberancias llamadas micro-aneurismas;
    \item NPDR moderada, donde los vasos sanguíneos se siguen debilitando, se producen más hemorragias, pudiendo provocar visión borrosa;
    \item NPDR severa, en esta etapa, los vasos sanguíneos están dañados, causando falta de oxígeno en la retina y en la formación de nuevos vasos;
    \item PDR(proloferative diabetic retinopathy) o retinopatía diabética proliferativa, donde los vasos sanguíneos anormales que crecen en la retina y en el vítreo. Estos vasos pueden sangrar y provocar desprendimiento de retina, provocando la pérdida de visión.
\end{itemize} 

En el caso de la retinopatía diabética, también se han desarrollado aplicaciones móviles como Ret-iN CaM~\cite{ret-in-cam}, la cual permite guardar imágenes de las retinas de los pacientes, obteniendo un historial para realizar un estudio de la evolución de su patología. Esto es debido al desarrollo de una lente que permite obtener fotos de la retina desde el dispositivo móvil.


Actualmente cuando un paciente acude a su médico de familia por pérdida de visión, el médico le da cita para el especialista para que éste decida si se tiene que hacer una imagen del fondo de la retina con un retinógrafo, un dispositivo que permite obtener imágenes del fondo de la retina con una gran resolución, para descartar la posibilidad una retinopatía diabética. 
\markboth{Introducción}{Introducción}
Una vez que el paciente acude a la consulta del especialista (lo que puede llevar varios meses de espera con las consiguientes complicaciones), se podrán descartar muchas personas que no tienen la enfermedad. Pero todas estas pruebas y los recursos invertidos suponen elevados costes para la sanidad pública y demoras en el tiempo de espera de los pacientes que realmente tengan un grado de retinopatía diabética.
 
Por estos motivos, si el médico de familia utiliza la lente comentada anteriormente, junto con el teléfono móvil, se podrían obtener las imágenes del fondo de la retina, con la suficiente calidad como para hacer un diagnóstico preliminar. De esta manera se podrían reducir también los costes económicos comentados anteriormente y sobre todo el tiempo de espera de los pacientes al reducir las consultas que se derivan al especialista.


Como solución a este problema, siguiendo con la tendencia de aplicar los avances informáticos a la medicina, se propone realizar una aplicación móvil, que permita al médico de familia hacer un estudio, a partir de una foto de la retina del paciente, utilizando una lente para el dispositivo móvil~\cite{d-eyecare-lente}.
Además, para facilitar la labor del médico, esta aplicación móvil, proporciona redes neuronales convolucionales ya entrenadas, con las que el médico obtendrá un primer análisis del paciente, sabiendo cuándo la imagen tiene una calidad aceptable. De esta forma, se evitaría una lentitud en el sistema sanitario, enviando al especialista a aquellas personas que en cuyo resultado haya algún grado de retinopatía diabética. 
 

\section{Estructura}
La memoria se puede organizar en los siguientes apartados:
\begin{itemize}
    \item \textbf{Introducción}: Sección que dispone el tema y resume el contenido del trabajo.
    \item \textbf{Objetivos del proyecto}: Apartado donde se explican los objetivos que se buscan conseguir.
    \item \textbf{Conceptos teóricos}: Parte del documento donde se exponen los conceptos claves del proyecto.
    \item \textbf{Técnicas y herramientas}: Lugar donde se explican las metodologías y herramientas utilizadas durante la realización del proyecto.
    \item \textbf{Aspectos relevantes}: Capítulo donde se recogen los datos más importantes del desarrollo.
    \item \textbf{Trabajos relacionados}: Comparación del trabajo realizado, con otros anteriores.
    \item \textbf{Conclusiones y líneas de trabajo futuras}: Apartado donde se expone de forma critica el trabajo realizado, indicando posibles mejoras a realizar en un futuro.
\end{itemize}

Además, se adjunta como anexo los siguientes apartados:
\begin{itemize}
    \item \textbf{Plan de proyecto}: En este apartado donde se explica cómo se ha organizado el proyecto, y estudios sobre la rentabilidad tanto económica como legal tiene el proyecto.
    \item \textbf{Requisitos}: Parte del documento donde se explican los requisitos funcionales y no funcionales que tiene el proyecto.
    \item \textbf{Diseño}: Capítulo donde se describe la fase de diseño del proyecto.
    \item \textbf{Manual del programador}: Manual donde se recogen los aspectos que necesitaría un programador, tanto como para seguir el trabajo, como para poder entender el funcionamiento.
    \item \textbf{Manual del usuario}: Manual donde se realiza una guía de usuario, con los pasos a seguir para un correcto funcionamiento de la aplicación.
\end{itemize}
