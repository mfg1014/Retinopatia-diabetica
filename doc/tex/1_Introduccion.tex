\capitulo{1}{Introducción}

Durante los últimos años, se ha aumentado exponencialmente el uso de los dispositivos móviles, este aumento se debe principalmente a la comodidad que proporciona respecto a otras tecnologías parecidas; siendo principalmente útiles para el uso de las redes sociales. 

Este aumento, tiene como consecuencia el fomento del desarrollo de aplicaciones móviles, puesto que sin aplicaciones para estos dispositivos, no hubiesen tenido tanto auge. Estas aplicaciones están destinadas a dos sistemas operativos principalmente, que son Android e iOS; donde el 81,3\% de los usuarios prefieren Android y el 13.4 \% iOS~\cite{android-iOS}.

Los móviles, no solo se han desarrollado en software, sino en hardware también, donde se han aumentado las características de procesador, cámara, RAM,...  Y con ello, los desarrolladores buscan crear aplicación que hagan uso de este hardware permitiendo muchas más funcionalidades; un uso que esta aumentando, es la integración de las inteligencias artificiales dentro de las propias aplicaciones.

Por otro lado, la salud y bienestar han sido temas importantes para el ser humano; y por lo tanto, también se han hecho aplicaciones móviles con este objetivo.

En el caso de la retinopatía diabética, se ha desarrollado una lente que permite obtener fotos desde el dispositivo móvil de la retina. Lo que ha llevado a aplicaciones móviles dedicadas a la retinopatía diabética, que permiten guardar imágenes de los pacientes, obteniendo un historial para realizar un estudio de la evolución de la patología en el paciente, como es el caso de ReTinCam.

La discapacidad visual afecta a más de 2200 millones de personas en el mundo, siendo las principales causas de pérdida de visión: degeneración macular relacionada con la edad, cataratas, retinopatía diabética, glaucoma y errores de refracción no corregidos.
Entre las personas con discapacidad visual, hay al menos 1000 millones de personas que tienen un deterioro moderado o grave de la visión\cite{oms-ceguera}.

En Estados Unidos, cada año la retinopatía diabética suma un 12\% de nuevos casos de ceguera \cite{porcentaje-afectados-eeuu}.

La retinopatía diabética~\cite{enfermedad-ocular-diabetica} es una complicación de la diabetes, que afecta al sistema ocular del ser humano. Es causada por la inflamación, escape o cierre de los vasos sanguíneos de la retina, pudiendo desarrollarse nuevos vasos sanguíneos a lo largo del tiempo. Esta anomalía, encabeza las causas de ceguera en los países desarrollados. Según la Organización Mundial de la Salud, hasta 1 millón de personas tienen ceguera debido a la diabetes~\cite{oms-diabetes}. 
Hay varios grados de la patología de la retinopatía diabética entre los que se encontrarían por orden de menor a mayor peligrosidad \cite{grados-retinopatia}: 
\begin{itemize}
    \item NPDR (Non-proloferative diabetic retinopathy), este grado se caracteriza por la ausencia de retinopatía;
    \item NPDR leve, donde los vasos sanguíneos empiezan a debilitarse, creando protuberancias llamadas micro-aneurismas;
    \item NPDR moderada, donde los vasos sanguíneos se siguen debilitando, se producen más hemorragias, pudiendo provocar visión borrosa;
    \item NPDR severa, en esta etapa, los vasos sanguíneos están dañados, causando falta de oxigeno en la retina y en la formación de nuevos vasos;
    \item PDR(proloferative diabetic retinopathy) o retinopatía diabética proliferativa, donde los vasos sanguíneos anormales que crecen en la retina y en el vítreo. Estos vasos pueden sangrar y provocar desprendimiento de retina, provocando la perdida de visión.
\end{itemize} 


En la actualidad, cuando un paciente acude a su médico de familia por perdida de visión, el médico determinaría si hacerle una cita para que el especialista le haga una imagen de la retina con el retinógrafo, para descartar la posibilidad de retinopatía diabética.

Este proceso, conlleva elevados costes para la sanidad publica, ya que implica un gran número de pruebas, junto con el número de tramites administrativos, en algunos casos, solo para descartar las personas que no tienen la enfermedad. Para el paciente supone un gran número de desplazamientos, que en algunos casos no se pueden permitir. Además de las horas de trabajo o lectivas que le supone al paciente.

Como solución a este problema, siguiendo con la tendencia de aplicar los avances informáticos a la medicina, se propone realizar una aplicación móvil, que permita al médico de familia hacer un estudio, haciendo una foto de la retina del paciente, utilizando una lente para el dispositivo móvil. 

Además, para facilitar la labor del médico, esta aplicación móvil, proporciona redes neuronales convolucionales ya entrenadas, con las que se obtendría un primer análisis, enviando al especialista a aquellas personas que en cuyo resultado haya algún grado de retinopatía diabética. 

De esta forma, al instalar la aplicación RetinAI creada con el objetivo de ser útil para el sistema sanitario, usando una interfaz sencilla para médicos sin experiencia previa en aplicaciones Android. Además, la aplicación facilita la creación de un informe médico ya que interpreta los resultados obtenidos de las imágenes, dando prioridad a aquellos casos que tengan más gravedad; y descartando aquellos casos que no tengan retinopatía diabética. Evitando una lentitud en el sistema sanitario debido a pacientes que no tienen esta complicación. 

Permitiendo a los especialistas centrar su atención en aquellos pacientes que realmente tienen retinopatía diabética.

\section{Estructura}
La memoria se puede organizar en los siguientes apartados:
\begin{itemize}
    \item \textbf{Introducción}: Apartado que dispone el tema y resume el contenido del trabajo.
    \item \textbf{Objetivos del proyecto}: Apartado donde se explican los objetivos que se buscan conseguir.
    \item \textbf{Conceptos teóricos}: Apartado del documento donde se exponen los conceptos claves del proyecto.
    \item \textbf{Técnicas y herramientas}: Apartado donde se explican las metodologías y herramientas utilizadas durante la realización del proyecto.
    \item \textbf{Aspectos relevantes}: Apartado donde se recogen los datos más importantes del desarrollo.
    \item \textbf{Trabajos relacionados}: Apartado donde se compara el trabajo realizado, con otros anteriores.
    \item \textbf{Conclusiones y lineas de trabajo futuras}: Apartado donde se expone de forma critica el trabajo realizado, indicando posibles mejoras a realizar en un futuro.
\end{itemize}

Además, se adjunta como anexo los siguientes apartados:
\begin{itemize}
    \item \textbf{Plan de proyecto}: Apartado donde se explica como se ha organizado el proyecto, y estudios sobre la rentabilidad tanto económica como legal tiene el proyecto.
    \item \textbf{Requisitos}: Apartado donde se explican los requisitos funcionales y no funcionales que tiene el proyecto.
    \item \textbf{Diseño}: Apartado donde se describe la fase de diseño del proyecto.
    \item \textbf{Manual del programador}: Apartado donde se recogen los aspectos que necesitaría un programador, tanto como para seguir el trabajo, como para poder entender el funcionamiento.
    \item \textbf{Manual del usuario}: Apartado donde se realiza una guía de usuario, con los pasos a seguir para un correcto funcionamiento de la aplicación.
\end{itemize}