\capitulo{1}{Introducción}

La discapacidad visual afecta a más de 2200 millones de personas en el mundo, siendo las principales causas de la perdida de visión: degeneración macular relacionada con la edad, cataratas, retinopatía diabética, glaucoma y errores de refracción no corregidos.


Entre las personas con discapacidad visual, hay al menos 1000 millones de personas que tienen un deterioro moderado o grave de la visión. En Estados Unidos, cada año la retinopatía diabética suma un 12\% de nuevos casos de ceguera.

La retinopatía diabética es una complicación de la diabetes, que afecta al sistema ocular del ser humano. Es causada por el daño a los vasos sanguíneos de la retina. Esta anomalía, encabeza las causas de ceguera en los países desarrollados. Según la Organización Mundial de la Salud, hasta 1 millón de personas tienen ceguera debido a la diabetes. 

Para determinar el grado que tiene un paciente de retinopatía diabética, se tiene que desplazar al hospital donde le hacen una imagen de la retina con un retinógrafo. Pero, para la sanidad publica tiene un coste elevado, y provoca que el paciente se tenga que desplazar a un centro hospitalario. 

Como solución a este problema, se propone realizar una aplicación móvil, que permita al paciente o al médico hacer una foto de la retina, utilizando una lente para el dispositivo móvil. Además, para facilitar la labor del médico, esta aplicación móvil, proporciona redes neuronales ya entrenadas, con los que se obtendría un primer análisis, proporcionando un estudio más intenso a aquellas personas que hayan dado algún grado de retinopatía diabética. 

