\apendice{Plan de Proyecto Software}

\section{Introducción}

\section{Planificación temporal}
Para la planificación temporal, cabe destacar que se ha seguido la metodología SCRUM, la cual no se ha podido seguir al 100\%, puesto que es una metodología planeada para equipos de trabajo, siendo este un proyecto personal de final de carrera. 

\begin{itemize}
    \item se programó el \textit{sprint} de dos semanas, donde, se podrían haber hecho de una semana, pero la metodología SCRUM recomienda de dos a cuatro.
    \item Durante, el  \textit{sprint} se planean unas tareas a terminar en el periodo de tiempo.
    \item Cada tarea tiene un objetivo y una estimación; se organizan en un tablero canvas con las siguientes columnas consideradas:
    \begin{itemize}
        \item "tareas a realizar" : Son las tareas planeadas para el sprint actual o próximos.
        \item "tareas en proceso" : Son aquellas tareas que se están realizando en el momento.
        \item "tareas terminadas" : Son las tareas que ya se han terminado, las cuales se comprueban al final del sprint si se han terminado correctamente, en tal caso se cierran.
        \item "tareas cerradas" : Son aquellas tareas que se han cerrado en \textit{sprints} anteriores, las cuales se han implementado correctamente en el producto final. 
    \end{itemize}
    \item Con los gráficos \textit{burndown}, se puede observar el proceso del proyecto de cada \textit{sprint}
    \item Como se ha comentado, al terminar el sprint,se realizaba una reunión personal, donde se comprobaba si se habían completado correctamente las tareas terminadas. En caso de no ser así, se añadía para el siguiente \textit{sprint}, además, se añadían las nuevas tareas a realizar en el siguiente.
\end{itemize}
Esta metodología se ha implementado usando la herramienta ZenHub, la cual permite hacer los apartados anteriores con facilidad.
Para asignar los pesos de la tarea, se utiliza una aproximación de la secuencia de fibonacci. Donde los valores dependen del tiempo y dificultad que llevará la tarea, esta estimación, a lo largo del proyecto va siendo más precisa, puesto que al principio, no se controlaban los tiempos que conllevarían las dificultades del proyecto.

Los \textit{sprints} que se realizaron se comentan a continuación.

\subsection{Sprint 0 - Anterior al 27/12/2022}
En este sprint inicial, del que no se poseen fechas, se realizaron diversas reuniones, en las cuales se explicaron en que iba a consistir el proyecto.

Durante este sprint, se estuvo investigando sobre aplicaciones Android con el mismo objetivo, sobre las que basar el proyecto. Encontrando de esta forma la aplicación Ret-iN CaM. La cual sirve para realizar estudios sobre los pacientes, de forma que se pueda ver la evolución.
Además, se realizaron búsquedas sobre redes neuronales ya creadas las cuales añadir a la aplicación. 

Como este sprint fue más un periodo de búsqueda de información, sin realizar ningún trabajo real, no se contabilizó el trabajo realizado.
\subsection{Sprint 1 - 27/12/2022 - 10/01/2023}
En este primer Sprint, se estuvo pensando con que herramienta realizar el proyecto. Teniendo como objetivo crear tanto el proyecto, como el LaTeX.

Para la aplicación móvil, se pensaba utilizar o Android Studio o Unity, de Android Studio ya se tenían conocimientos de uso, pero de Unity no, por ello, se realizo un proyecto ''Hola Mundo'' en Unity, para ver la complejidad; al ver que la herramienta estaba más destinada a aplicaciones gráficas, se decidió realizar la aplicación en Android Studio, creando el proyecto.

Para la realización del documento, se pensó hacer de forma local, utilizando MiKTeX, pero por facilidad de uso de los tutores, quienes preferían el uso de OverLeaf, que les permitía corregir de forma concurrente el documento, se utilizó esta.

Como este primer sprint se estaba empezando, y además era Navidad, se olvidó mover las tareas de ZenHub a terminadas, y la gráfica que proporciona ZenHub es incorrecta. 

\imagen{burndown/primerSprint.png}{Sprint 1.}
\subsection{Sprint 2 - 10/01/2023 - 24/01/2023}
En el segundo Sprint se planeo hacer el menú principal de la aplicación, para ir haciendo la interfaz de usuario y a su vez, se hizo una investigación sobre que colores relaciona el ser humano con la retinopatía diabética.
Para hacer el menú, primero se hizo un boceto a mano, por ese motivo, se puso un valor alto de la estimación.
Por otro lado, se investigó sobre como implementar las redes neuronales en Android Studio, e ir añadiendo apartados al documento LaTeX.

De estas tareas, no se pudieron terminar ni la de investigar como implementar una red neuronal en Android Studio, ni añadir el apartado en concreto del LaTeX. Aunque si que se avanzó, SCRUM no permite dividir la tarea en parte terminada y parte que no, por lo que se va a considerar que no se han realizado.

El gráfico burndown se vería incorrectamente, debido al inconveniente ocurrido en el primer sprint, arreglando este defecto, se vería de esta forma:
\imagen{burndown/segundoSprint.png}{Sprint 2.}

\subsection{Sprint 3 - 24/01/2023 - 07/02/2023}
En el tercer Sprint, se añadieron las tareas no terminadas en el anterior, y se añadió también, el terminar el boceto inicial de la aplicación.
Al iniciar el sprint, se pensó en que la aplicación tenia que tener nombre y un logo, pero se programó para futuros sprints porque ya se tenían varias tareas para este.
Además, no se pudo avanzar mucho en la introducción del LaTeX porque era necesario aclarar una duda presencialmente, y la reunión se hizo a finales del sprint.

Como se iba avanzando en las tareas simultáneamente, tanto terminar el boceto de la aplicación y buscar información de la implementación, se terminaron el mismo día.
\imagen{burndown/tercerSprint.png}{Sprint 3.}
\subsection{Sprint 4 - 07/02/2023 - 21/02/2023}
Para el cuarto sprint, se planeó realizar como una investigación sobre como se podría incluir la aplicación en la play store, y sobre como integrar tensorflow en Android Studio.

Al investigar sobre la play store, se dio cuenta que era necesario realizar un pago de 25\$, por ese mismo motivo, se preguntó a los tutores, quienes dijeron que no hacia falta subir la aplicación a ninguna tienda.

Por otro lado, se terminaron las demás tareas, cumpliendo los objetivos del sprint. Y aunque se podría haber eliminado la tarea de investigación de la play store, el tiempo gastado en la investigación se había realizado, por lo que no se elimino.

El burndown queda de la siguiente manera:
\imagen{burndown/cuartoSprint.png}{Sprint 4.}

\subsection{Sprint 5 - 21/02/2023 - 07/03/2023}
Para este sprint, como no había tareas atrasadas, se añadieron la creación del logo y el nombre de la aplicación, de la investigación anterior de como implementar una red neuronal en Android Studio, el caso concreto de una red neuronal .h5; realizar cambios en la interfaz para añadir y quitar funcionalidad. Y añadir una base de datos; para lo que se pensó realizar un conjunto de clases java, las cuales simulaban la base de datos, y con un constructor, inicializar los datos.

Debido a que no se me da muy bien en el diseño artístico, se me ocurrió la idea de utilizar la herramienta DALL·E2 para un diseño preliminar, una vez se proporcionó un resultado aceptable, se editó el icono eliminando ruido, y cambiando los colores.

Debido a las diversas actividades designadas en este sprint, no se pudo realizar la base de datos.
\imagen{burndown/quintoSprint.png}{Sprint 5.}

\subsection{Sprint 6 - 07/03/2023 - 21/03/2023}
En el sexto sprint, se añadieron realizar el apartado de trabajos relacionados del documento LaTeX, y probar una integración de una red neuronal básica.
Ambas tareas se terminaron, y se avanzo en la creación de la base de datos; pero no se terminó debido a que se querían añadir más datos, los cuales solo se crearon un médico, un paciente y un informe.

Sin contar que se podría haber considerado como la tarea terminada, el gráfico burndown, seria este:
\imagen{burndown/sextoSprint.png}{Sprint 6.}

\subsection{Sprint 7 - 21/03/2023 - 04/04/2023}
En este sprint, se termino la base de datos añadiendo otro par de pacientes y médicos, y se planearon hacer una eliminación de bugs, debido al modo oscuro, y que no guardaba bien datos entre las distintas actividades. Y también se pensó hacer el apartado "añadir técnicas y herramientas", pero se avanzó, sin poder acabar la tarea. Quedando para la siguiente actividad.

\imagen{burndown/septimoSprint.png}{Sprint 7.}



\section{Estudio de viabilidad}

\subsection{Viabilidad económica}

\subsection{Viabilidad legal}


