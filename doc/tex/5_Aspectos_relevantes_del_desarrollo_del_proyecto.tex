\capitulo{5}{Aspectos relevantes del desarrollo del proyecto}

En este apartado se recogen los aspectos más importantes. Explicando las decisiones tomadas del proyecto, y las consecuencias que suponen, comentando los errores y como se solucionaron.

\section{Inicio}

Una vez se me explicó la idea que se buscaba con este proyecto, me gustó la idea de poder hacer una aplicación que pudiera servir al sistema sanitario publico.

En este inicio, empecé a imaginarme como podría ser la aplicación haciendo bocetos mentales de sus distintas pantallas, como era muy efímero, se hizo una búsqueda de aplicaciones similares, para comprobar alguna interfaz a añadir, donde se encontró la aplicación RetinCam. Con una idea más formada, se hicieron los bocetos iniciales de la aplicación.

\section{Metodologías}

Como ya se ha comentado anteriormente, se decidió utilizar una metodología SCRUM, no siguiéndose al completo, puesto que los equipos de desarrollo en esta metodología están compuesto de 3 a 9 personas, a su vez, hay reuniones que no se han podido realizar, entre otras cosas. Pero, con esta metodología se ha buscado que el proyecto tuviese una metodología ágil.

Un fallo cometido con respecto esta metodología ha sido que los martes cambiaba de sprint a las 8 de la mañana, y la revisión para finalizar el sprint, se realizaba los martes por la mañana, haciendo que algunas veces las tareas cambiasen de sprint cuando no debían, y por tanto, la reunión para comenzar el sprint se realizaba con el sprint comenzado.

Las caracteristicas agiles de este proyecto son:
\begin{itemize}
    \item Los sprints planeados han tenido una duración de 2 semanas, entregando un incremento al final de cada uno.
    \item Se han realizado reuniones, tanto para finalizar el sprint, como para el inicio del siguiente, teniendo en consideración el fallo comentado anteriormente.
    \item Las tareas que se planeaban para un sprint, se estimaban y priorizaban en un tablero canvas, tanto físicamente como con la herramienta ZenHub. Aunque con el paso de los sprints, se dejo de hacer físicamente.
    \item Para comprobar el progreso del proyecto, se ha utilizado los gráficos burndown, que aunque los ofrece ZenHub, se han realizado a mano, por el fallo comentado.
\end{itemize}

Al principio del proyecto se planeaba utilizar otras metodologías como \textit{Test-Driven-Development} o como \textit{Data-driven testing}, junto con pruebas automáticas, vistas durante el año académico en la asignatura Validación y pruebas, para comprobar como las implementaciones que se iban realizando en el proyecto no se veían afectadas entre ellas y que se implementaban correctamente. Pero al implementar las distintas interfaces, se decidió hacer pruebas manuales, las cuales proporcionan una mayor comprensión de la interacción que tiene el usuario con la aplicación, buscando siempre que el usuario entienda el sistema.

