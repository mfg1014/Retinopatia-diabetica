\capitulo{2}{Objetivos del proyecto}

Los objetivos del proyecto se pueden dividir en 3 apartados, los cuales se verán a continuación. 
\section{Objetivos generales}
\begin{itemize}
    \item Desarrollar una aplicación Android, que permita realizar un estudio de la retinopatía diabética sobre los pacientes.
    \item Agilizar el sistema sanitario, haciendo que los médicos de familia puedan dar un diagnostico preliminar a partir de los datos proporcionados.

\end{itemize}
\section{Objetivos técnicos}
\begin{itemize}
    \item Desarrollar una aplicación Android con soporte API 21, siendo compatible con dispositivos Android 5.0 (Lollipop) y superiores.
    \item Hacer uso de Material 3 como fuente para que la aplicación sea más accesible al usuario.
    \item Hacer uso de Gradle para la automatización de la construcción de software.
    \item Hacer uso de GitHub como herramienta para alojar proyectos.
    \item Hacer uso de Git como herramienta de control de versiones.
    \item Hacer uso de la metodología SCRUM haciendo uso a su vez de la herramienta ZenHub.
    \item Hacer uso de GitKraken para hacer uso de una interfaz que facilite las acciones con el repositorio en GitHub.
    \item Hacer uso de pruebas, automáticas y manuales de forma que no se produzcan errores de ejecución.
    \item Hacer uso de Python para convertir un modelo keras con formato ``.h5'' a un modelo tensorflow Lite con formato ``.tflite''.
    \item Hacer uso de keras, en Python, para realizar una red neuronal convolucional.
    \item Hacer uso de TensorFlow Lite, para poder proporcionar los resultados en una aplicación Android.
    
\end{itemize}
\section{Objetivos personales}
\begin{itemize}
    \item Hacer uso de los conocimientos vistos durante la carrera.
    \item Aprender cómo desarrollar aplicaciones, en las que hay que implementar una red neuronal.
    \item Realizar una aplicación para dispositivos Android, que se pueda ejecutar en la mayoría de estos teléfonos, facilitando de esta forma la integración en el sistema sanitario.
    
\end{itemize}