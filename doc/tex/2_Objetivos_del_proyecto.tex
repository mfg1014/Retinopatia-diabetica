\capitulo{2}{Objetivos del proyecto}

Los objetivos del proyecto se pueden dividir en 3 apartados, los cuales se verán a continuación. 
\section{Objetivos generales}
\begin{itemize}
    \item Proporcionar a los centros sanitarios una aplicación, que permita diferenciar cuándo un paciente posee o no retinopatía diabética.
    \item Agilizar el sistema sanitario, haciendo que los médicos de familia puedan dar un diagnóstico preliminar a partir de los datos proporcionados.
    \item Desarrollar un modelo de red neuronal que permita identificar la calidad de la imagen seleccionada, de forma que el médico sepa si la imagen tiene suficiente calidad, o tiene que repetir la imagen.

\end{itemize}
\section{Objetivos técnicos}
\begin{itemize}
    \item Desarrollar una aplicación Android con soporte API 21, siendo compatible con dispositivos Android 5.0 (Lollipop) y superiores.
    \item Hacer uso de Material 3 como fuente para que la aplicación sea más accesible al usuario.
    \item Usar la herramienta Gradle para la automatización de la construcción de software.
    \item Utilizar GitHub como herramienta para alojar proyectos.
    \item Utilizar Git como herramienta de control de versiones.
    \item Emplear la metodología SCRUM haciendo uso a su vez de la herramienta ZenHub.
    \item Hacer uso de GitKraken para hacer uso de una interfaz que facilite las acciones con el repositorio en GitHub.
    \item Crear pruebas, de forma que no se produzcan errores de ejecución.
    \item Convertir un modelo keras con formato ``.h5'' a un modelo tensorflow Lite con formato ``.tflite''.
    \item Realizar una red neuronal convolucional utilizando un modelo keras ya entrenado.
    \item Utilizar TensorFlow Lite, para poder proporcionar los resultados en una aplicación Android.
    
\end{itemize}
\section{Objetivos personales}
\begin{itemize}
    \item Hacer uso de los conocimientos vistos durante la carrera.
    \item Aprender cómo desarrollar aplicaciones, en las que hay que implementar una red neuronal.
    \item Realizar una aplicación para dispositivos Android, que se pueda ejecutar en la mayoría de estos teléfonos, facilitando de esta forma la integración en el sistema sanitario.
    
\end{itemize}